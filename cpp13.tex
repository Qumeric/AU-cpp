\section{20 мая}

\subsection{Conditional variables}

Код с прошлой лекции на новый лад.
\begin{minted}{c++}
int get() {
  m.lock();
  while(q.empty()) {
    m.unlock();
    m.lock();
  }
  int s = q.pop();
  m.unlock();
  return s;
}

void put(int s) {
  m.lock();
  while(q.size() == capacity()) {
    m.unlock();
    m.lock();
  }
  q.push(s);
  m.unlock();
}
\end{minted}

Проблема данного когда в том, что ждать разблокировки таким образом (в цикле) неэффективно. Кроме того, он слегка громоздок.
На помощь приходят conditional variables.
\begin{minted}{c++}
void put(int s) {
  // Совмещает возможности mutex и lock_guard
  unique_lock<mutex> l(m);
  // Спим, пока не разбудят, проверяем условие
  // Если нет, то спим дальше, иначе работаем
  not_full.wait(l, [this]{return g.size() != capacity});
  q.push();
  // Разбудить один consumer
  not_empty.notify_one();
}

int get() {
  unique_lock<mutex> l(m);
  not_empty.wait(l, [this] { return !g.empty() });
  int s = q.pop();
  not_full.notify_one();
}
\end{minted}

wait внутри устроен примерно так:
\begin{minted}{c++}
while(!condition) {
  wait(l);
}
\end{minted}

% Вопрос к экзамену: почему именно так?

\subsection{Initializer list}
Можно инициализировать большинство структур данных с помощью фигурных скобок.
Для того, чтобы добавить поддержку в свой класс, делаем так:

\begin{minted}{c++}
vector::vector(initializer_list<T>& l) {
initializer_list<T>::iterator it = l.begin();
  T t = *it;
  // Тут создаем объект
}

\end{minted}
Компилятор сначала создает массив в статической памяти, затем вызывает метод.

\subsection{unordered\_set, unordered\_map}

\begin{minted}{c++}
template<>
struct hash<PhoneEntry> {
  size_t operator()(const PhoneEntry& o) {
    return hash<int>()(num) + hash<string>()(s);
  }
};

class PhoneEntry {
  string s;
  int num;
  bool operator==(consta PhoneEntry& o);
  friend class std::hash<PhoneEntry>;
};
\end{minted}

\subsection{Множественное наследование}

Суть состоит в том, что можно наследоваться от нескольких классов одновременно. 

Внимание! Существует распространенное мнения, что данную возможность вообще никогда не стоит использовать.

Существует ромбовидное наследование. В этом случае объекты не всегда можно неявно привести к типу какого-либо предка. В таких неоднозначных местах нужно использовать \cc{static_cast}, либо \textbf{виртуальное наследование}.

\begin{minted}{c++}
struct A {
    int foo() { return 1; }
};
class B: public virtual A {};
class C : public virtual A {};
class D : public B, public C {};
int main () {
    D d;
    cout << d.foo();
}
\end{minted}

Если убрать ключевое слово virtual, то метод \cc{foo()} не может быть определён однозначно и в результате не будет доступен как объект класса D и код не скомпилируется.
