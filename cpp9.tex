\section{15 апреля 2016 г.}

\subsubsection{Lambda функции}
\begin{minted}{c++}
find_if(Iterator first,
        Iterator last,
        Functor operator);
\end{minted}
Предположим, что мы хотим найти первое положительное значение в контейнере. Пишем так:
\begin{minted}{c++}
struct PositiveFinder {
  bool operator()(int v) { // Для простоты int
    return v > 0;
  }
};

vector<int> v(100);
auto f = find_if(begin(v), end(v), PositiveFinder());
\end{minted}
Что здесь плохо? Громоздкий синтаксис, для каждой цели нужно писать свой функтор (ну или хотя бы функцию). Кроме того, это засоряет namespace.

В $C++11$ предложено такое решение с использование лямбд(анонимных классов).
\begin{minted}{c++}
auto f = find_if(begin(v), end(v), [](int n) {return n > 0});
// Можно и с параметрами
f = find_if(begin(v), end(v), [threshold](int n){return n > threshhold});
// Варианты
[=] // Все локальные текущего scope передаются по значению
[&] // Тоже самое, но по ссылке
[=var1, &var2] // Одно по значению, одно по ссылке (= можно не писать)
\end{minted}
Захват переменных (\cc{[]}) называется замыканием (closure).

\subsection{vardiadic templates(и functions)}
Это есть даже в $C$.
\begin{minted}{c++}
// Все делается написанием одной-единственной функции
printf("%s, %d", s, d);
printf("%d", d);

// Здесь определены макросы va_start, va_arg, va_end
#include<stdarg.h>

int min(int n, ...) { // Произвольное число аргументов
  va_list args; // Список аргументов
  va_start(args, n); // Аргументы начинаются после n
  int  = 0, m = INF;
  while(i++ < n) { // n - число аргументов
    int arg = va_arg(args, int); // Извлекаем очередной аргумент типа int
    m = min1(m, arg);
  }
  // Если аргументы выделяются с помощью malloc, то надо очистить память
  va_end(args);
  return m;
}
\end{minted}
Как это устроено внутри? Может быть много вариантов, один из ниx такой:
\begin{minted}{c++}
  typedef unsigned char* va_list;
  // va_start делает примерно это:
  args = &n + sizeof(n);
  // va_arg делает примерно это: %FIXME подумать
  T var = T(*args);
  args += sizeof(T);
\end{minted}

В C++ развили эту идею.
\begin{minted}{c++}
struct Fact {
  static const int val = n*Fact<n-1>::val;
};

template<>
struct Fact<0> {
  static const int val = 1;
};

int main() {
  int v = Fact<5>::val; // Вычисляется во время компиляции
}
\end{minted}

\begin{minted}{c++}
template<typename T, typename ...Args>

T min(T n, Args ...rest) {
  return min(n, min(rest...));
}
template<typename T>
T min(T a, T b) {
  if (a < b) return a;
  return b;
}
\end{minted}
