\documentclass[a4paper,12pt]{article}
\pagestyle{empty}
\usepackage{color}
\usepackage[utf8]{inputenc}
\usepackage[T2A]{fontenc}
\usepackage[russian]{babel}

\usepackage{enumitem}
\usepackage{etex}
\usepackage{amsmath}
\usepackage{amsthm}
\usepackage{amssymb}
\usepackage{verbatim}
\usepackage{minted}
\usepackage{qtree}

\DeclareMathOperator{\Ima}{Im}
\DeclareMathOperator{\rk}{rk}

\newtheorem{theorem}{Теорема}
\newtheorem{definition}{Определение}
\newtheorem{proposal}{Предложение}
\newtheorem{lemma}{Лемма}
\newtheorem{remark}{Замечание}
\newtheorem{example}{Пример}

\renewcommand{\O}{\mathcal{O}}
\newcommand{\R}{\mathbb{R}}
\newcommand{\Q}{\mathbb{Q}}
\renewcommand{\C}{\mathbb{C}}
\newcommand{\eps}{\varepsilon}
\newcommand{\Ra}{\Rightarrow}
\newcommand{\ra}{\rightarrow}
\newcommand{\Lra}{\Leftrightarrow}
\newcommand{\Sum}{\sum \limits}
\newcommand{\cc}{\mintinline{c++}}

\def\changemargin#1#2{\list{}{\rightmargin#2\leftmargin#1}\item[]}
\let\endchangemargin=\endlist 

\title{Конспект по C++}
\author{Черепанов Валерий}
\date{25 марта 2016 г.}

\begin{document}
\maketitle
\part{Лекция 0}

\section{Итераторы внутри STL}

\begin{minted}{c++}
template<class Iter>
void sort(Iter p, Iter q);

list<int> l;
vector<int> v;
sort(l.begin(), l.end());
sort(v.begin(), v.end());
\end{minted}

\subsection*{Проблемы}
\begin{enumerate}
    \item Знаем итератор, но не знаем, например, тип элементов вектора.
    \item Не знаем, что умеет итератор (например, может ли он в random access?). Поэтому большинство операций с итератором обарачиваем в библиотечные функции
    \begin{minted}{c++}
advance(Iter& it, int n);
distance(Iter& it1, Iter& it2);
    \end{minted}
\end{enumerate}

\subsection*{Решения}
Как решена проблемы в STL?
\begin{minted}{c++}
template<class T>
class vector {
    T *array;
    class Iterator {
        typedef value_type T; // Решение первой проблемы
        // В sort пишем typename Iter::value_type var; 
        typedef interator_category ra_iterator; // Решение второй проблемы
    };
};
\end{minted}

Как делать ``\cc{if}'' по типу? Перегрузкой!
\begin{minted}{c++}
template<class Iter>
void advance (Iter it, int n) {
    typename Iter::iterator_category ite;
    advance_impl(it, n, ine);
}
template<class Iter>
void advance_impl(Iter& it, int n, ra_iterator it) {
    it += n;
}
template <class Iter>
void advance_impl(Iter int n, int n, bidi_iterator it) {
    int i = 0;
    if (n > 0) {
        while(i < n) {
            ++it;
            ++i;
        }
    }
    if (n < 0) {
        while(i > n) {
            --it;
            --i;
        }
    }
}
\end{minted}
Используется полиморфизм времени компиляции. % FIXME what?

\section{Iterator traits}
Хотим делать примерно то же самое, но не для итераторов, а для указателей.
Проблема:
\begin{minted}{c++}
template<class Iter>
void sort(Iter p, Iter q) {
    Iter::value_type;
}
\end{minted}
Если вызовем sort от двух указателей, то получим compilation error.

Решение проблемы:
\begin{minted}{c++}
template<class Iter>
class iter_traits {
    typedef value_type Iter::value_type;
    typedef iterator_category Iter::iterator_category;
}
vector<int>;
typename iter_traits<vector<int>::iterator>::value_type a;
\end{minted}

Кажется, мы ничего на самом деле не решили, а просто написали какую-то чушь. Но на самом деле это не так, нужно лишь воспользоваться специализацией шаблонов!

\begin{minted}{c++}
template<typename Iter*> // специализация для указателей
class iter_type {
    typedef value_type Iter;
};
\end{minted}
\cc{Iter:iterator_category} $\ra$ \cc{iter_traits<Iter>::iterator_category}

\end{document}
