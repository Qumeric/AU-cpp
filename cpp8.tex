\section{22 апреля 2016 г.}

\subsubsection{auto и decltype}
\begin{minted}{c++}
vector<int>::iterator it = vec.begin();
auto it = vec.begin(); // автоматический вывод типа

int a;
int& ref = a;
auto b = ref; // int, & снимается
auto& c = ref; // int

template<typename T, typename V>
void multiply(vector<T>& a, vector<T>& b) {
  auto tmp = a[i]*b[i]; // Error. Не знаем, какой будет тип
}

template<class T, class V>
declype(T(0)*V(0)) multiply(const T& a, const V& b) {
  decltype(a*b) product_type; // Такой же тип, как у operator*(T, V);
  product_type tmp = a*b;
  return tmp;
}

template<class T, class V>
// Просто auto нас не поймет (хотя в C++14 может)
auto multiply(const T& a, const V& b) -> decltype(a*b) {
  return a*b;
}
\end{minted}

\subsection{move семантика}
Используется, если класс(структура) имеет указатели. В таком случае, если мы хотим ``переместить'' объект, то можно лишь переместить их, а не копировать что-либо.
В $C++$ все выражения делятся на lvalue и rvalue. Грубо говоря, lvalue "--- то, что можно писать слева от ``$=$'', а rvalue "--- все остальное.
\begin{minted}{c++}
template<class T>
// Будут выполнены лишние действия в виде копирований (например, из a в tmp)
void swap(T& a, T& b) {
  T tmp = a;
  a = b;
  b = tmp;
}
\end{minted}
Решение: конструктор и оператор присванивания.
\begin{minted}{c++}
  // %FIXME UNDONE
\end{minted}
