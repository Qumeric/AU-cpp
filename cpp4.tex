\section{18 марта 2016 г.}

\subsection{Исправление некоторых неточностей}

\subsubsection{reserve}
Для вызова \cc{vector.reserve} не требуется вызвать какие-либо конструкторы (даже по умолчанию). Там используется placement new, то есть просто выделяется кусок памяти с помощью new и sizeof, а потом пишется \cc{new(adress) Obj} где \cc{adress} "--- это выделенная память.

\subsubsection{string}
Раньше string старались быть ленивыми и при создании старались сделать ссылку на уже созданное. Но плохо было то, что мы могли с легкостью получить линию при обращении с помощью \cc{[]} (строка копировалась в новую память, чтобы ее можно было модифицировать), да к тому же мы могли инвалидировать итератор.

\subsubsection{Манипуляторы потоков}
Манипуляторы вроде \cc{hex} как правило просто выставляют необходимый флаг с помощью \cc{setf}. Для их использования перегружен оператор \cc{<<}, он принимает поток и указатель на функцию.

\subsection{STL}

\subsubsection{Кое-что про map}
Как уже говорилось, \cc{map} "--- это по сути \cc{set} пар. Оператор \cc{[]} у \cc{map} создает новый элемент (вызывая стандартный конструктор), если мы обратились по несуществующему ключу, потому что он не знает, хотим ли мы лишь получить значение по ключу или изменить его.

\subsubsection{Кое-что про set}
\begin{minted}{c++}
class Person {
    string name;
    int age;

    bool operator <(const Person& p) {
        return name < p.name;
    }
}
\end{minted}

Если мы создадим \cc{set<Person>}, то элементы будут отсортированы по полю \cc{name}. Но мы можем захотеть сортировать по \cc{age}. Как этого добиться? На помощь приходят функторы!

\begin{minted}{c++}
struct by_age {
    bool operator()(const Person& p1, const Person& p2) {
        return p1.age < p2.age;
    }
}

set<Person, by_age> s; // Используем так
\end{minted}
Внутри это устроено примерно так:
\begin{minted}{c++}
template<typename T, class comparator>
class set {
    insert(...) {
        if (comparator()(n1, n2)) { ... } // Анонимный объект
    }
}
\end{minted}
Если мы не передаем в \cc{set} второй шаблонный параметер, то используется стандартный функтор \cc{less}. Он выглядит примерно так:
\begin{minted}{c++}
template<typename T>
struct less {
    bool operator()(const T& t1, const T& t2) {
        return t1 < t2;
    }
}
\end{minted}

\begin{remark}
    \cc{multimap} и \cc{multiset} "--- это тоже деревья, но обычно в узле хранится список элементов.
\end{remark}

\subsubsection{\cc{algorithm}}
Итераторы "--- обертки над указателями. Их идея заключается в том, что алгоритм может работать с разными структурами данных, поддерживающими одинаковые операции (от структуры требуется лишь предоставить итераторы и методы работы с ними).

Некоторые алгоритмы STL:
\begin{enumerate}
    \item \cc{swap(T& a, T& b), max(T& a, T& b)}
    \item \cc{count[_if](It a, It b, const T& x)} \\
        \cc{x} "--- значение в \cc{count}, функция или функтор в \cc{count_if} (можно считать сложные функции или экономить время на сравнение).
    \item \cc{equal(It a, It b, It it) // Сравнивает [a, b) с [it, ...)} 
    \item \cc{sort, min_element, nth_element, reverse, ...}
\end{enumerate}

\subsubsection{Чем хороши функторы?}
Тем, что можно передавать параметры!
\begin{minted}{c++}
class Finder {
private:
  int what;
public:
  Finder(int w):what(w){}
  bool operator()(int n) {
    return n > what;
  }
};
// v - какой-то контейнер
find_if(begin(v), end(v), Finder(473));
\end{minted}

\subsubsection{Чем плохи итераторы?}
У них слишком маленький базовый интерфейс (\cc{++, --, *, ->}), поэтому они ``из коробки'' не подходят для адекватной реализации многих алгоритмов (даже для бинарного поиска). Кроме того, мы не знаем тип объекта итератора, а он может нам понадобиться.
Для решения этих проблем были придуманты \cc{iterator_traits}.

Как часть решения первой проблемы алгоритмы обарачивают некоторые обращения к итераторам используя функции \cc{distance} и \cc{advance}.

С точки зрения перемещения итераторы бывают:
\begin{enumerate}
    \item Forward
    \item Bidirectional
    \item Random Access
\end{enumerate}
