\section{13 мая}
\subsection{timed\_mutex}
Еще один тип мьютексов "--- \cc{timed_mutex}. Он используется, когда есть функция (в данном примере \cc{f}), которая работает с внешиними данными.
\begin{minted}{c++}
while(true) {
  // Ждем 100 милисекунд, вдруг мьютекс освободится?
  if(mutex.try_lock_for(timeout) {
    f();
    mutex.unlock();
  } else {
    g(); // f() выполнять нельзя, займемся чем-нибудь другим
  } 
}
\end{minted}

\subsection{Пример приложения с потоками}
Типы взаимодействия между потоками:
\begin{description}
  \item[readers/writers] Потоки могут писать, читать, либо делать и то, и другое.
  \item[producers/consumers] Одни потоки производят задания и кладут их в очередь, другие их выполняют.
\end{description}

\begin{minted}{c++}

my_queue q;
mutex m;
void push(int i) {
  flag = false;
  while(flag != true) {
    m.lock();
    if (!q.full()) {
      flag = true;
    }
    m.unlock();
  }
  lock_guard l(m);
  q.push(i);
}

void pop() {
  // Тут проверка, что в очереди что-то есть,
  // По аналогии с проверкой на полноту в push
  lock_guard l(m);
  q.pop();
}

int main() {
  vector<consumer> cs;
  vector<producer> ps;
  for(consumer &c: cs) {
    c.init(q); c.start()
  }
  for(producer &p: ps) {
    p.init(q); p.start()
  }
}
\end{minted}

Получилось не очень удобно. В следующей лекции будут рассмотрены conditional variables, которые позволяют решить это удобное.
