\section{15 апреля 2016 г.}

\subsection{C++11}

\subsubsection{Избавляемся от define}

\paragraph{nullptr}
Раньше было так:
\begin{minted}{c++}
// Было
int *x = 0;
int *p1 = NULL; // #define NULL (void*)0;

void f(long l) { }
void f(char *s) { }

int main() {
  f(0L); // OK
  char s[] = "Hello";
  f(s); // OK
  f(NULL);  // Ambiguity
  f(0);     // Ambiguity
}
\end{minted}
В C++ эту проблему постарались решить с помощью нового ключевого слова \cc{nullptr}. Такой ``указатель'' неявно кприводиться к указателю любого другого типа.

\paragraph{static\_assert}
Раньше был только обычный \cc{assert}.
\begin{minted}{c++}
  // Если boolvar = false, то runtime error.
  assert(boolvar);
\end{minted}
Теперь появился \cc{static_assert}, который выполняется на стадии компиляции.
\begin{minted}{c++}
  // Второй аргумент - сообщение об ошибке
  static_assert(sizeof(int)>=4, "This program could not work");

  template<typename T, size_t size>
  class static_array {
    T array[size];
    static_array() {
      static_assert(size > 0, "Empty arrays are prohibited");
    }
  }
  static_array<int, 100> a; // OK
  static_array<int, 0> b;   // Error
\end{minted}

\paragraph{Type alias}
Раньше:
\begin{minted}{c++}
// Так можно
typedef map<string, list<int>> ml_t;
f(ml_t a);

// А вот так уже нет
typedef map<string, list<T>> ml_t;
f(ml_t<int> a);
\end{minted}
C++11 спешить на помощь!
\begin{minted}{c++}
template<typename T>
using mt_t=map<string, list<T>>
\end{minted}

\paragraph{constexpr}
Было:
\begin{minted}{c++}
size_t square(int x) a {
  return x*x;
}
const int sz = cl;
int a[sz]; // OK
static_array<int, sz> sa; // OK
int a[square(3)]; // Error
static_array<int, square(3)> sa; // Тоже error
\end{minted}
Решение:
\begin{minted}{c++}
// constexpr говорит, что функция при определенных
// условиях может использоваться в константных выражениях
constexpr size_t square(int x) a {
  return x*x;
}
\end{minted}

\subsubsection{Декораторы}

\paragraph{default и delete}
\begin{minted}{c++}
class T {
  int v = 0; // Такая инициализация - тоже фича C++11
public:
  // default говорит, что в любом случае нужно создать метод по умолчанию
  T() = default;
  // delete говорит, что не стоит создавать метод по-умолчанию
  T& operator= (const T&) = delete;
  T(const T&) = delete;
}
\end{minted}

\paragraph{override}
Проблема:
\begin{minted}{c++}
class Base {
public:
  virtual void f(int) const;
  virtual int g() const;
  void n(int) const;
}

class Derived:public Base {
  void f(int) const;
  virtual int g(int); // Забыли const
  void n(int) const;
}

const Base *b = new Derived();
b->f()
b->g(); // Base::g(), так как сигнатуры не совпадают
b->n();
\end{minted}
Решение "--- \cc{override}. Это слово говорит компилятору, что функция что-то перекрывает. ``Под капотом'' компилятор просто проверяет, есть ли такая виртуальная функция в базовом классе.
\begin{minted}{c++}
class Derived:public Base {
  void f(int) const override; // OK
  virtual int g(int) override; // Error
  void n(int) const override; // OK
}
\end{minted}

\paragraph{Конструкторы}
\begin{minted}{c++}
class MyNumber {
  MyNumber(char* s): A(atoi(s)) { } // Вызываем один конструктор из другого
  MyNumber(int i) { };
}

MyNumber("64");
MyNumber(64);
\end{minted}
